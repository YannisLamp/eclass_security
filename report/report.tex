\documentclass{article}
\usepackage{graphicx}
\usepackage[english,greek]{babel}
\usepackage[utf8x]{inputenc}
\usepackage{amsmath}
\usepackage{relsize}
\usepackage{enumerate}
\usepackage[parfill]{parskip}
\usepackage{graphicx}
\usepackage{listings}

\makeatletter
\renewcommand*\env@matrix[1][*\c@MaxMatrixCols c]{%
  \hskip -\arraycolsep
  \let\@ifnextchar\new@ifnextchar
  \array{#1}}
\makeatother

\begin{document}

\title{\vspace{-3.5cm}\textbf{Προστασία και Ασφάλεια Υπολογιστικών Συστημάτων \\ΕΑΡΙΝΟ 2018\\ \textlatin{Project \#}1}}
\author{Λάμπρου Ιωάννης \\1115201400088\\\\ Στεφανίδης - Βοζίκης Κωνσταντίνος \\1115201400192}

\maketitle
\section*{\textlatin{Defense}}
\subsection*{\textlatin{SQL Injection}}
Το \textlatin{openeclass}, στην μορφή που το παραλάβαμε, πριν τροποποιήσουμε τον κώδικά του, 
βασιζόταν κυρίως στην λειτουργία των \textlatin{magic quotes} για άμυνα έναντι \textlatin{SQL Injections},
αλλά επιπλέον είχαν δημιουργηθεί οι συναρτήσεις \textlatin{escapeSimple()} και \textlatin{autoquote()} οι
οποίες έκαναν έλεγχο αν τα \textlatin{magic quotes} ήταν ενεργοποιημένα στον \textlatin{server}, και, αν
όχι, τότε εφάρμοζαν τις \textlatin{mysql\_real\_escape\_string()} ή \textlatin{mysql\_escape\_string()} και
\textlatin{addslashes()} στα ορίσματά τους αντίστοιχα. Δυστυχώς οι συναρτήσεις αυτές όμως δεν
χρησιμοποιούνταν σε πάρα πολλά μέρη όπου μια εξωτερική μεταβλητή
έμπαινε σε \textlatin{SQL query}. Ακόμα, οι σελίδες του \textlatin{admin} ήταν απροστάτευτες, δικαιολογημένα.\\
Εμείς, για προστασία, αφού τα \textlatin{prepared statements (libraries MySQLi, PDO)} δεν επιτρέπονται στα
πλαίσια της εργασίας, επιλέξαμε την επόμενη καλύτερη λύση, την χρήση της συνάρτησης
\textlatin{mysql\_real\_escape\_string()}.
Η συνάρτηση αυτή εφαρμόστηκε σε κάθε μεταβλητή η οποία γινόταν \textlatin{append}
σε \textlatin{sql query string}, σε μερικές περιπτώσεις αντικαθιστώντας τη χρήση των υπολοίπων παραπάνω
συναρτήσεων. Ακόμη να σημειωθεί πως η λειτουργικότητα της εφαρμογής δεν άλλαξε, αφού επιτρέπονται ακριβώς τα ίδια \textlatin{inputs} με προηγουμένως, κάνοντας όμως καλό \textlatin{escaping} στις ερωτήσεις στη βάση.


\subsection*{\textlatin{XSS}}
Αρχικά, το \textlatin{openeclass} είχε πολύ λίγη προστασία ενάντια σε \textlatin{XSS}
(ειδικά στα \textlatin{modules} που κληθήκαμε να προστατεύσουμε), με εξαίρεση τις σελίδες του
\textlatin{admin} οι οποίες ήταν πολύ καλά προστατευμένες, (είχε οριστεί ακόμα και η συνάρτηση
\textlatin{q()} ως συντόμευση της \textlatin{htmlspecialchars()}) Για την προστασία της εφαρμογής, τώρα πια
κάθε μεταβλητή η τιμή της οποίας εμφανιζόταν στον τελικό \textlatin{html} κώδικα, (γινόταν
\textlatin{append} στο \textlatin{\$tool\_content}) φιλτράρεται πρώτα μέσω της κλίσης της συνάρτησης
\textlatin{htmlspecialchars()} είτε προέρχεται από ερώτηση στη βάση ή κάποιο αρχείο, είτε από τον χρήστη της
εφαρμογής μέσω \textlatin{GET} ή \textlatin{POST}. Πρέπει να τονιστεί το γεγονός ότι ποτέ τα
\textlatin{inputs} δεν φιλτράρονται κατά την αποθήκευση, αλλά μόνο όταν πρόκειται να εμφανιστούν σε
\textlatin{html} κώδικα.


\subsection*{\textlatin{CSRF}}
Όσον αφορά την άμυνα για τις επιθέσεις \textlatin{CSRF}. Ποστατευθήκαμε βάζοντας \textlatin{CSRF tokens}
σε κάθε φόρμα ή οποία προκαλεί αλλαγές στην σελίδα. Σε κώδικα \textlatin{PHP} ή άμυνα μοιάζει κάπως
έτσι:\\
\includegraphics[scale=0.5]{csrf}\\
Πρώτα ενεργοποιούμε ενα \textlatin{token} και κατόπιν το βάζουμε ως \textlatin{hidden field} στην
φόρμα που μας ενδιαφέρει. Ο έλεγχος εγκυρότητας γίνεται ως εξης:\\
\textlatin{if (isset(\$\_POST['hide']) and \$\_POST['hide'] == 0 and !empty(\$\_POST['token']) and
(strcmp(\$\_SESSION['token'], \$\_POST['token']) === 0))}\\
Η ίδια λογική άμυνας υπάρχει σε κάθε φόρμα της ιστοσελίδας (η οποία προκαλεί αλλαγές).
Επίσης, επειδή δεν υπάρχει άμυνα έναντι επιθέσεων \textlatin{CSRF} σε \textlatin{GET requests}, διάφορα
\textlatin{GET} που άλλαζαν την σελίδα αλλάχθηκαν σε \textlatin{POST} ώστε να γίνει η ίδια άμυνα.
Ένα παράδειγμα είναι η διαγραφή χρήστη από την σελίδα του \textlatin{admin}.

\subsection*{\textlatin{RFI}}
Τέλος, για προστασία κατά επιθέσεων \textlatin{RFI}, δημιουργήσαμε ένα αρχείο \textlatin{.htaccess} με το \textlatin{option Options -Indexes}, έτσι ώστε να μην είναι δυνατόν να μάθει κανείς τα ονόματα των \textlatin{directories} όπου αποθηκεύονται οι εργασίες ή τα ονόματα των αρχείων που ανεβαίνουν μέσω της ανταλλαγής αρχείων. (Τα οποία ονόματα δημιουργούνται τυχαία από την εφαρμογή). Μια καλή πρακτική, που όμως δεν ήταν δυνατή στα πλαίσια της εργασίας, είναι όλα τα αρχεία που προέρχονται από χρήστες να αποθηκεύονται στον \textlatin{server} σε τελείως ξένα \textlatin{directories} με την εφαρμογή έτσι ώστε να μην μπορεί να τα κάνει \textlatin{request} κάποιος. Τέλος, στην εφαρμογή δεν παρατηρήθηκε κάποιο \textlatin{dynamic include} αρχείων αναβασμένων από χρήστη. 








\section*{\textlatin{Attack}}








\subsection*{\textlatin{SQL Injection}}
Για επίθεση \textlatin{SQL Injection}, σε περίπτωση που βρισκόταν κάποιο \textlatin{vulnerability}, θα υπήρχε η δυνατότητα να βρεθεί το \textlatin{hashed password} του \textlatin{drunkadmin}, ή να μάθουμε το όνομα του \textlatin{directory} που δημιουργείται για την αποθήκευση των εργασιών ή τα ονόματα των αρχείων που έχουμε στείλει για ανταλλαγή αρχείων, έτσι ώστε να προσπαθήσουμε να τα κάνουμε \textlatin{request} και να επιτύχουμε \textlatin{RFI}. Ακόμη μια καλή στρατιγική θα ήταν να μετατρέψουμε τον χρήστη μας σε \textlatin{admin}, αλλάζοντας το \textlatin{userid} του \textlatin{admin} σε κάτι εκτός από 1 και κάνοντας το δικό μας 1, αφού η εφαρμογή θεωρεί ότι ο χρήστης με \textlatin{userid} 1 είναι ο \textlatin{admin}. Σημειώνεται πως δεν γνωρίζουμε το δικό μας \textlatin{userid} αλλά θα μπορούσαμε είτε να το μάθουμε, αν έχουμε βρει \textlatin{SQL Injection vulnerability}, είτε να μαντέψουμε, αφού δεν είναι και πολλές περιπτώσεις, ανάλογα και με τους συνολικούς χρήστες της εφαρμοφής. Δυστυχώς η αντίπαλη ομάδα ήταν καλά προστατευμένη έναντι σε \textlatin{SQL Injection} και δεν βρέθηκαν \textlatin{vulnerabilities}. Με την προσθήκη, για παράδειγμα, σε ένα \textlatin{keyword} στις σελίδες για \textlatin{search} ή σε ένα \textlatin{url GET} του \textlatin{string ' AND '1' = '1} τότε και δεν εμφανίζονταν τα ίδια αποτελέσματα με πριν, ενώ όταν η παραπάνω συμβολοσειρά έμπαινε σε \textlatin{input box} τότε εμφανιζόταν αυτούσια, πράγματα που μας έκαναν να καταλάβουμε πως η αντίπαλη εφαρμογή έκανε σωστό \textlatin{escaping}. 





\subsection*{\textlatin{XSS}}
Αντίθετα, η αντίπαλη εφαρμογή είχε πολλές ευπάθειες σε \textlatin{XSS attacks}. Aρχικά, στην ανταλλαγή αρχείων, το όνομα αλλά και η περιγραφή του αρχείου μπορούσαν να πάρουν την τιμή \textlatin{script} και να εκτελεστούν κανονικά.
Επίσης στο \textlatin{calendar}, μπορεί να περαστεί και να εκτελεστεί \textlatin{script} μέσω \textlatin{GET}. Στο \textlatin{module} περιοχές συζητήσεων, αν και ο \textlatin{xinha editor} είχε αυτόματη προστασία από \textlatin{XSS}, υπήρχε ένα \textlatin{option} το οποίο επέτρεπε εισαγωγή κώδικα \textlatin{html} χωρίς φιλτράρισμα το οποίο δεν είχε προστατευτεί. Ακόμα, στην σελίδα όπου γίνεται ορισμός νέου συνθηματικού, αν εισαχθεί ένα \textlatin{script} την θέση του ονόματος χρήστη μαζί με ένα \textlatin{valid e-mail}, τότε και εμφανίζεται μια σελίδα λάθους, η οποία εκτελεί το \textlatin{script}. Αφού η σελίδα χρησιμοποιεί \textlatin{POST} όμως, θα ήταν αναγκαίο να συνδυαστεί με μία επίθεση \textlatin{CSRF}. Η τηλεσυνεργασία και τα στοιχεία του χρήστη ήταν καλά προστατευμένα. 

\subsection*{\textlatin{CSRF}}
Η σελίδα ήταν γενικά πολύ ευάλωτη σε επιθέσεις \textlatin{CSRF}. Η πρώτη
επίθεση που δοκιμάστηκε και ήταν επιτυχής ήταν η \textlatin{malicious} εισαγωγή
μαθήματος απο τον διαχειριστή ο οποίος ανυποψίαστος πάτησε ένα λίνκ που του στειλαμε με
\textlatin{email}. Άλλες επιθέσεις με \textlatin{CSRF} δεν πραγματοποιήθηκαν καθώς μετά
κάναμε ολοκληρωτικό \textlatin{deface} της σελίδας οπότε και το \textlatin{CSRF} ήταν πιο
ανίσχυρο από τις άλλες επιθέσεις. Παρόλα αυτά είχαν βρεθεί ευπάθειες και είχαν σχεδιαστεί
επιθέσεις σε μια σωρεία λειτουργιών όπως η διαγραφή εργασίας, διαγραφή μαθήματος, διαγραφή
χρήστη, δημιουργία ανακοίνωσης, δημιουργία εργασίας σε μάθημα.



\subsection*{\textlatin{RFI}}
Σχετικά με \textlatin{RFI}, ανεβάσαμε \textlatin{php} αρχεία τόσο μέσω των εργασιών όσο και μέσω της ανταλλαγής αρχείων τα οποία έγιναν αποδεκτά, αλλά χωρίς να μπορούμε να τα κάνουμε \textlatin{request}, μη ξέροντας τα ονόματα του καταλόγου και των αρχείων, αντίστοιχα. (τα \textlatin{Indexes} ήταν \textlatin{disabled}).


\section*{\textlatin{Defacement} και \textlatin{hashed drunkadmin password}}
Τελικά, καταφέραμε να κάνουμε \textlatin{deface} και να μάθουμε το \textlatin{hashed password} της αντίπαλης αφαρμογής χρησιμοποιώντας έναν συνδυασμό από τους παραπάνω τρόπους επίθεσης, εκμεταλλευόμενοι τις παραπάνω ευπάθειες. Αρχικά, εκγαταστήσαμε ένα \textlatin{script} στις περιοχές συζητήσεων του μαθήματος των αντιπάλων, το οποίο και παρέπεμπε σε έναν \textlatin{cookie stealer} σε δικό μας server, και στείλαμε \textlatin{links} στον \textlatin{drunkadmin} για να μπει στο μάθημα αυτό (η εφαρμογή χρησιμοποιεί μια \textlatin{global} μεταβλητή η οποία αλλάζει τιμή όταν ο χρήστης επιλέγει ένα μάθημα) και στην περιοχή συζητήσεων όπου είχε εγκατασταθεί το \textlatin{script}. Με αυτόν το τρόπο, λάβαμε το \textlatin{cookie} του \textlatin{admin} \textlatin{0327c65e36a8df3d73e3c973d244e33f} και το χρησιμοποιήσαμε για να μπούμε στην εφαρμογή ως διαχειριστές. Από αυτό το σημείο έιχαμε πολλές δυνατότητες, όπως να ενεργοποιήσουμε \textlatin{modules} για μάθημα που σίγουρα θα ήταν απροστάτευτα, αλλά επιλέξαμε να πάρουμε τον κωδικό και το \textlatin{port} για σύνδεση στη βάση των αντιπάλων.\\
\includegraphics[scale=0.4]{term}\\ 
Έτσι, μετά από σύνδεση στη βάση, μάθαμε το \textlatin{hashed drunkadmin password} αλλά και το \textlatin{directory name} όπου και αποθηκεύονταν οι εργασίες, έστι ώστε να μπορέσουμε να κάνουμε \textlatin{request} μια δική μας "εργασία" η οποία είναι ένα αρχείο \textlatin{php} το οποίο παίρνει ένα \textlatin{input}, εκτελεί την εντολή \textlatin{shell\_exec()} με αυτό το \textlatin{input}, και εμφανίζει το \textlatin{output}. (Το αρχείο \textlatin{sur name} στην παρακάτω εικόνα)\\\\
\includegraphics[scale=0.4]{term2}\\

Τελικά, μπορέσαμε να κάνουμε \textlatin{request} το αρχείο αυτό και να τρέχουμε εντολές κατευθείαν στον \textlatin{server} της εφαρμογής, δίνοντάς μας την δυνατότητα να αντικαταστήσουμε το \textlatin{index.php} με ένα δικό μας.

\end{document}
