\documentclass{article}
\usepackage{graphicx}
\usepackage[english,greek]{babel}
\usepackage[utf8x]{inputenc}
\usepackage{amsmath}
\usepackage{relsize}
\usepackage{enumerate}
\usepackage[parfill]{parskip}

\makeatletter
\renewcommand*\env@matrix[1][*\c@MaxMatrixCols c]{%
  \hskip -\arraycolsep
  \let\@ifnextchar\new@ifnextchar
  \array{#1}}
\makeatother

\begin{document}

\title{\vspace{-3.5cm}\textbf{Αλγόριθμοι και Πολυπλοκότητα}\\Εργασία 1 \\ Λύσεις}
\author{Λάμπρου Ιωάννης}

\maketitle
\section*{1.}
%\begin{enumerate}
%\item
Ο αλγόριθμος \textlatin{A} περιγράφεται από την αναδρομική εξίσωση: $T_A(n) = 5T(\frac{n}{2}) + n$\\\\
Αν εφαρμόσουμε το \textlatin{Master Theorem}, βλέπουμε ό,τι $f(n) = n, a=5, b=2$\\\\
Και: \  $n^{\log_b{a}} = n^{\log_2{5}} = n^{2.321}$ και
$f(n) = O(n^{\log_b{a-\epsilon}}) $ για $\epsilon = 1,321$.\\\\
Άρα σύμφωνα με τη πρώτη περίπτωση, $T_A(n) = $Θ$(n^{2.321})=O(n^{2.321})$\\\\

Ο αλγόριθμος \textlatin{B} περιγράφεται από την αναδρομική εξίσωση: $T_B(n) = 2T(n-1) + 1$\\\\
Αν την χειριστούμε όπως το παράδειγμα της αναδρομικής εξίσωσης των πύργων του \textlatin{Hanoi}, με την μέθοδο των αθροιζομένων παραγόντων, λαμβάνουμε:\\\\
$T_B(n) = 2^nT_B(0) + (1 + ... + 2^{n-1}) = 2^n0 + \frac{2^n-1}{2-1}$ (Αν βέβαια $T_B(0) = 0$)\\\\
Το οποίο ισούται με $2^n-1$ Άρα και $T_B(n)=$Θ$(2^n)=O(2^n)$\\\\

Ο αλγόριθμος \textlatin{C} περιγράφεται από την αναδρομική εξίσωση: $T_C(n) = 9T(\frac{n}{3}) + n^2$\\\\
Αν εφαρμόσουμε το \textlatin{Master Theorem}, βλέπουμε ό,τι $f(n) = n^2, a=9, b=3$\\\\
Και: \  $n^{\log_b{a}} = n^{\log_3{9}} = n^2$ = f(n)\\\\
Άρα σύμφωνα με την δεύτερη περίπτωση, $T_C(n) = $Θ$(n^{2}\log{n})=O(n^{2}\log{n})$\\\\\\

\section*{2.}
Για να λύσουμε το πρόβλημα αυτό θα μπορούσαμε να χρησιμοποιήσουμε μια παραλλαγή της \textlatin{Merge Sort}, δηλαδή αντί να ταξινομήσουμε μία ακολουθία (πίνακα) \textlatin{n} στοιχείων,
θα πρέπει να ταξινομήσουμε μια ακολουθία (πίνακα) από \textlatin{k} ταξινομημένες ακολουθίες, με \textlatin{n} στοιχεία η καθεμία.\\\\
Χρησιμοποιώντας τη στρατηγική 'διαίρει και βασίλευε':\\\\
\textlatin{\textbf{Divide}}: Διαίρεση του πίνακα \textlatin{k} ταξινομημένων στοιχείων σε δύο υποπίνακες $\frac{k}{2}$ στοιχείων (πινάκων) ο καθένας.\\\\
\textlatin{\textbf{Conquer}}: Αναδρομική ταξινόμηση των δύο υποπινάκων (που αποτελούνται από ταξινομημένους πίνακες) με τη χρήση του αλγορίθμου μας.\\\\
\textlatin{\textbf{Combine}}: Συγχώνευση των δύο ταξινομημένων υποπινάκων.\\\\
Στην ουσία η αναδρομή τελειώνει όταν φτάσουμε στο σημείο να έχουμε έναν υποπίνακα που περιέχει μόνο δύο ταξινομημένους υποπίνακες, όπου και τους συγχωνεύουμε και τους επιστρέφουμε.
Λόγω της ομοιότητας της λύσης με τον αλγόριθμο \textlatin{Merge Sort}, η χρονική πολυποκότητά της θα είναι $O(kn\log{kn})$, αφού συνολικά τα στοιχεία θα είναι $kn$.


\section*{3.}
\begin{enumerate}[(a)]%for small alpha-characters within brackets.
\item
Σωστό. Το $\omega{()}$ είναι αυστηρότερο, άρα και η $f(x)$ θα είναι και  Ω$(g(n))$ (Αφού το Ω είναι υπερσύνολο του ω)
\item
Λάθος. Για $n > 1$, πάντα η$(10n^2 + kn + c)$ θα είναι μεγαλύτερη ή ίση από τη $(4n^2 + 5n - 9)$ $(4n^2 + 5n - 9) = O(10n^2)$ 
\item
Σωστό. Ξέρουμε από τις διαφάνειες πως $\log{n!} = O(n\log{n})$ (Άσκηση 4, σελ 15), ενώ $\log{n!} = \log{1} + \log{2} + ... + \log{n} >= \\\\
 \log{\frac{n}{2}} + \log{\frac{n}{2}+1} + ... + \log{n} = \log{\frac{n}{2}}*\frac{n}{2} = \frac{n\log{n}}{2}-\frac{n\log{2}}{2} =>\\\\
\log{n!} >=  \frac{n\log{n}}{2}-\frac{n\log{2}}{2}$, Άρα και $\log{n!}$ $=$Ω$(n\log{n})$. Αφού  το $\log{n!}$ είναι και Ο και Ω του $(n\log{n})$, τότε θα είναι και Θ
\item
Λάθος. Ξέρουμε πως $f(n) + g(n) = $Ω$(min(g(n),f(n)))$ ενώ $f(n) + g(n) = O(max(g(n),f(n)))$ άρα ισχύει μόνο στην περίπτωση που $f(n) = g(n)$
\item
Σωστό. Αν πάρουμε το όριο, $ lim_{n\to\infty} \frac{n+2\sqrt{n}}{n\sqrt{n}} = lim_{n\to\infty} \frac{\sqrt{n}+2}{n} = 0$, άρα είναι O
\item
Σωστό. Ξέρουμε ότι το όριο $ lim_{n\to\infty} (g(n) - f(n)) = -\infty$. Αν πάρουμε το όριο, $ lim_{n\to\infty} \frac{2^{g(n)}}{2^{f(n)}} =  lim_{n\to\infty} 2^{g(n)-f(n)}$ το οποίο και θα κάνει μηδέν, λόγω του αρχικού ορίου. Άρα και το $2^{f(n)}$ θα είναι Ω$(2^{g(n)})$ άρα και ω.
\item
Σωστό. Αποδείχτηκε στο προηγούμενο ερώτημα.
\item
Σωστό. Έστω $f(x) = $ω$(g(x)$, τότε θα πρέπει για $x>=x_0 f(x) > g(x)$. Ομοίως, αν $f(x) = $ο$(g(x)$, τότε θα πρέπει για $x>=x_0 f(x) < g(x)$ Έτσι, βλέπουμε ότι τα δύο αυτά ενδεχόμενα είναι ξένα.

\end{enumerate}


\section*{4.}
Οι κλάσεις θα είναι (από μικρότερη σε μεγαλύτερη πολυπλοκότητα):
\begin{enumerate}
\item
$10^{100} $
\item
$\log{\log{n}} , \log{\sqrt{\log{n}}}$
\item
$\log{n}$
\item
$\mathlarger{\sum\limits_{i=0}^n \left(\left(\frac{2}{3}\right)^i\right)^n    },   \mathlarger{\sum\limits_{k=1}^n \sqrt[k]{k} }, 2^{\log^2{n}} $
\item
$ n\log{n},  \log{n^n}, \log{n!} $
\item
$ (\sqrt{2})^{\log{n}}$
\item
 $4^{\log{n}}$  
\item
 $n^{\log{n}} $
\item
$\mathlarger{\sum\limits_{k=1}^n \frac{n}{k}}, 8n^2  $
\item
$\binom{n}{2} $
\item
$2^n$
\item
$n!,  (n-1)! $ \\\\
\end{enumerate}
Άρα, τελικά:\\\\
$10^{100} <  \log{\log{n}} < \log{n} < 2^{\log^2{n}} <  n\log{n} <  (\sqrt{2})^{\log{n}} < 4^{\log{n}} < n^{\log{n}} < 8n^2 < \binom{n}{2} < 2^n < n!$

\section*{5.}
\renewcommand{\arraystretch}{2.5}
\begin{center}
\begin{tabular}{ c c|c|c|c|c|c| } 
 Α & Β & Ο & ο & Ω & ω & Θ \\
\hline 
 $\mathlarger{ \left( \frac{10}{11} \right)^n }$ & $\mathlarger{\log{n} }$ & Ναι & Ναι & Όχι & Όχι & Όχι\\
\hline 
 $\mathlarger{\sum\limits_{k=0}^n \binom{n}{k} }$ & $\mathlarger{2^{n/2} }$ & Ναι & Ναι & Όχι & Όχι & Όχι \\
\hline
$ \mathlarger{n^{\log{c}}} $ & $\mathlarger{c^{\log{n}}} $ & Ναι & Ναι & Όχι & Όχι & Όχι \\
 \hline
$ 8^{\log{n}} $ & $ n^2 $ & Όχι & Όχι & Ναι & Ναι & Όχι\\
\hline
$ n^k $ & $ c^n $ & Ναι & Ναι & Όχι & Όχι & Όχι\\
\hline
\end{tabular}
\end{center}


\section*{6.}
\begin{enumerate}[(a)]%for small alpha-characters within brackets.\
\item
Σε αυτήν την άσκηση θα εκμεταλλευτούμε το γεγονός ότι για να είναι ένα χρώμα το κυρίαρχο χρώμα ενός πίνακα, τότε και αν χωρίσουμε τον πίνακα σε τέσσερις ίσους υποπίνακες (μπορούμε διότι ο πίνακας είναι τετράγωνος) τότε και θα πρέπει σίγουρα αυτό το χρώμα να είναι το κυρίαρχο χρώμα σε τουλάχιστον έναν από αυτούς. Άρα, χωρίζουμε λοιπόν τον πίνακα σε τέσσερις μικρότερους και καλούμε τον αλγόριθμο για καθέναν από αυτούς. (Ο αγόριθμος θα επιστρέψει όχι μόνο το κυρίαρχο χρώμα, αλλά και τον αριθμό των κελιών εκείνου του χρώματος). Έτσι, για κάθε κυρίαρχο χρώμα που βρίσκεται, (μέγιστο τέσσερα χρώματα) μπορούμε μετά να κάνουμε σειριακή καταμέτρηση για αυτό το χρώμα και στους υπόλοιπους τρεις υποπίνακες (αν αυτό το χρώμα δεν ήταν το κυρίαρχο σε αυτούς) μέχρι να μετρήσουμε πάνω από $\frac{n*n}{2}$ κελιά τέτοιου χρώματος, οπότε και τελειώνει ο αλγόριθμος. Αν είναι κυρίαρχο ένα χρώμα σε πάνω από έναν υποπίνακα, τότε βέβαια δεν χρειάζεται να ψάξουμε. Άρα:\\\\

\textlatin{\textbf{Divide}}: Διαίρεση του πίνακα σε τέσσερις υποπίνακες διαστάσεων $\frac{n}{2}*\frac{n}{2}$.\\\\
\textlatin{\textbf{Conquer}}: Αναδρομική αναζήτηση κυρίαρχου χρώματος σε καθέναν από αυτούς.\\\\
\textlatin{\textbf{Combine}}: Για κάθε ένα κυρίαρχο χρώμα που βρέθηκε, τότε σειριακή καταμέτρηση των κελιών αυτού του χρώματος στους υπόλοιπους τρεις υποπίνακες για να δούμε αν ο αριθμός τους ξεπερνάει το $\frac{n*n}{2}$\\\\

Η αναδρομή βέβαια σταματάει όταν βρεθούμε σε πίνακα με τέσσερις πλάκες, οπότε απλά μετράμε τα χρώματα.\\

\item
Η αναδρομική εξίσωση θα είναι της μορφής  $T(n) = 4T(\frac{n}{4}) + 4*3n^2$\\
(Αφού στη χειρότερη περίπτωση θα βρούμε τέσσερα διαφορετικά κυρίαρχα χρώματα και θα χρειαστεί τέσσερις φορές να ψάξουμε για αυτά στους υπόλοιπους τρεις πίνακες.\\\\
Αν εφαρμόσουμε το \textlatin{Master Theorem}, βλέπουμε ό,τι $f(n) = 12n^2, a=4, b=4$\\\\
Και:  $n^{\log_b{a}} = n^{\log_4{4}} = n$ και σύμφωνα με την τρίτη περίπτωση που υπερισχύει η $f(x)$, τότε και η πολυπλοκότητα του αλγορίθμου θα είναι Θ$(n^2) = O(n^2)$.\\

\item
Εφόσον δεν υπάρχει περιορισμός στον αριθμό των χρωμάτων που χρησιμοποιούνται, τότε στην χειρότερη περίπτωση ο πίνακας θα περιέχει $n*n$ αριθμό χρωμάτων, ένα χρώμα ανά κελί.
Έτσι ο εξαντλητικός αλγόριθμος θα χρειαστεί να κάνει $n*n$ αναζητήσεις σε $n*n$ αριθμό κελιών, και άρα θα έχει πολυπλοκότητα ίση με $O(n^4)$, πολύ χειρότερη από την $O(n^2)$.

\end{enumerate}
\section*{7.}
\begin{enumerate}
\item
 Aν θεωρήσουμε ότι η πράξη $sum = sum + 1$ εκτελείται σε χρόνο $O(1)$, τότε και η πολυπλοκότητα του αλγορίθμου θα είναι:\\

$\frac{n}{2}*(2n^2)*(4n) = 5n^4 = O(n^4)$\\\\
Αφού οι επαναλήψεις είναι σταθερές, (δεν εξαρτάται η μία από την άλλη).\\\\
\item
Aν θεωρήσουμε ότι οι πράξεις μέσα στα βαθύτερα \textlatin{for loops} εκτελούνται σε χρόνο $O(1)$  τότε και η πολυπλοκότητα του αλγορίθμου θα είναι η μεγαλύτερη από τις τρεις παρακάτω περιπτώσεις:\\\\
Για τις γραμμές 1-8 η πολυπλοκότητα θα είναι:\\\\
$n*(2n^2)*(\frac{n}{2}) = n^4 = O(n^4)$\\\\
Αφού οι επαναλήψεις είναι σταθερές, (δεν εξαρτάται η μία από την άλλη).\\\\
Για τις γραμμές 9-16 η πολυπλοκότητα θα είναι:\\
Σε αυτή την περίπτωση ο αριθμός των επαναλήψεων είναι μεταβλητός, άρα:\\\\
$\sum\limits_{i=1}^n \sum\limits_{j=1}^{i*n} \frac{j}{2}   = \sum\limits_{i=1}^n ni\frac{ni+1}{4} =  \frac{n^2}{4} \sum\limits_{i=1}^n i^2 +  \frac{n}{4}\sum\limits_{i=1}^n i\
=  \frac{n^2}{4} * \frac{(n(n+1)(n+2)}{6} +  \frac{n}{4} * \frac{(n(n+1))}{2}$ \\\\\\
$ = \frac{n^5+3n^4+2n^3}{24} + \frac{n^3+n^2}{8}$\\\\
Άρα και η πολυπλοκότητά τους θα είναι $O(n^5)$\\\\

Για τις γραμμές 17-25 η πολυπλοκότητα θα είναι:\\\\
Σε αυτή την περίπτωση ο αριθμός των επαναλήψεων είναι μεταβλητός, άρα: (στο τρίτο \textlatin{for loop} θα είναι σταθερές)\\\\
$\sum\limits_{i=0}^{n^2-1} ((n^2-2-ι)*n^2) = n^2( \sum\limits_{i=0}^{n^2-1} n^2 - \sum\limits_{i=0}^{n^2-1}2 - \sum\limits_{i=0}^{n^2-1} ι)\\\\\\
= n^2((n^2)(n^2) - 2n^2 -  \frac{(n^2(n^2+1))}{2} ) = n^6 - 2n^4 - \frac{n^6 + n^4}{2} $\\\\
Άρα και η πολυπλοκότητά τους θα είναι $O(n^6)$

Τελικά η πολυπλοκότητα του αλγορίθμου θα είναι $O(n^6)$ 



\end{enumerate}
%\end{enumerate}
\end{document}